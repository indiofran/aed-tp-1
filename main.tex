\documentclass[10pt,a4paper]{article}

\input{AEDmacros}
\usepackage{caratula} % Version modificada para usar las macros de algo1 de ~> https://github.com/bcardiff/dc-tex
\usepackage{dirtytalk}

% Custom macros
\newcommand{\ciudad}{\ensuremath{\langle \mathsf{String} , \ent \rangle}}
% Custom macros
\newcommand{\matrizEnt}{\TLista{\TLista{\ent}}}

\titulo{Trabajo práctico 1: Especificación y WP}
\subtitulo{En búsqueda del camino}

\fecha{\today}

\materia{Algoritmos Y Estructura de Datos}
\grupo{Grupo QUKDCOGWWYZWJZQBIYWM}

\integrante{Gomez Salaverri, Francisco Nicolas}{550/15}{fsalaverri@dc.uba.ar}
\integrante{Barberón, Federico Joaquín}{112/24}{jfedericobarberonj@gmail.com}
\integrante{Fuertes Vila, Isabel}{474/92}{isblfv@gmail.com}
\integrante{Apellido, Nombre4}{004/01}{email4@dominio.com}
% Pongan cuantos integrantes quieran

% Declaramos donde van a estar las figuras
% No es obligatorio, pero suele ser comodo
\graphicspath{{../static/}}


% Document
\begin{document}

\maketitle

\section{Problemas}
A continuación, se presentan los procedimientos solicitados en el enunciado del trabajo, junto con las definiciones, funciones auxiliares y predicados que forman parte de la especificación del problema 'En búsqueda del camino' de este trabajo práctico.

\subsection{Definiciones}
Las siguientes definiciones se utilizan para proporcionar mayor claridad y legibilidad al trabajo.

Ciudad:   \ciudad


\subsection{Procedimientos}


\begin{proc}{grandesCiudades}{\In ciudades : \TLista{Ciudad}}{ \TLista{Ciudad}}
	\requiere{|ciudades| \neq 0}
	\asegura{\paraTodo[unalinea]{i}{\ent}{enRango(i, ciudades) \yLuego ciudades[i][1] > 50000 \implica ciudades[i] = res[i]}}
\end{proc}

\begin{proc}{sumaDeHabitantes}
	{\In menoresDeCiudades : \TLista{Ciudad}, \In mayoresDeCiudades : \TLista{Ciudad}}{ \TLista{Ciudad}}

	\requiere{ \\
		|menoresDeCiudades| = |mayoresDeCiudades| \land \\
		todasCiudadesDistintas(menoresDeCiudades) \land \\
		tienenMismosNombres(menoresDeCiudades, mayoresDeCiudades) \\
	}
	\asegura{\\
		|res| = |menoresDeCiudades| \land \\
		tienenMismosNombres(res, menoresDeCiudades) \land \\
		\paraTodo[unalinea]{c}{Ciudad}{c \in res \Then \\ c_1 = cantidadHabitantes(c_0, menoresDeCiudades) + cantidadHabitantes(c_0, mayoresDeCiudades)} \\
	}
\end{proc}

\begin{proc}{hayCamino}{\In distancias : \matrizEnt, in desde: \ent, hasta: \ent}{ \bool}
	\requiere{ 0 \leq desde,hasta < \longitud{distancias} \wedge esMatrizCuadrada(distancias)}
	\asegura{res = \True \Iff existeCamino(distancias, desde, hasta)}
\end{proc}

\begin{proc}{cantidadCaminosNSaltos}{\Inout conexion : \TLista{\TLista{\ent}}, \In n: \ent}{}
	\requiere{expresionBooleana1}
	\asegura{expresionBooleana2}
\end{proc}


\begin{proc}{caminoMínimo}
  {\In origen:\ent , \In destino: \ent , \In distancias : \TLista{\TLista{\ent}}}
  {\TLista{\ent}}

  \requiere{
    origen \neq destino \wedge hayCamino(distancias, origen, destino)
  }

  \asegura{
    res = s \Iff \\
    \existe[unalinea]{s}{\TLista{\ent}}{
      secuenciaEnMatriz(s, distancias) \yLuego 
      caminoValido(s, distancias, origen, destino) \implicaLuego \\
      \paraTodo[unalinea]{s'}{\TLista{\ent}}{
        (secuenciaEnMatriz(s', distancias) \yLuego 
         caminoValido(s', distancias, origen, destino))
        \implicaLuego \\
        \sum_{i=0}^{|s|-2} distancias[\;s[i]\;][\;s[i+1]\;]
        \leq 
        \sum_{i=0}^{|s'|-2} distancias[\;s'[i]\;][\;s'[i+1]\;]
      }
    }
  }
\end{proc}


\subsection{Auxiliares}
\aux{cantidadHabitantes}{nombre: \str, ciudades: \TLista{Ciudad}}{\ent}{\\
	\displaystyle\sum_{i = 0}^{|ciudades| - 1} IfThenElse(ciudades[i]_0 = nombre, ciudades[i]_1, 0)
}

\subsection{Predicados}
\pred{todasCiudadesDistintas}{ciudades: \TLista{Ciudad}}{
	\paraTodo[unalinea]{i}{\ent}{enRango(i, ciudades) \implicaLuego \neg \existe[unalinea]{j}{\ent}{(enRango(j, ciudades) \land i \neq j) \yLuego ciudades[i]_0 = ciudades[j]_0}}
}

\pred{tienenMismosNombres}{ciudades1, ciudades2: \TLista{Ciudad}}{
	\paraTodo[unalinea]{c1}{Ciudad}{c1 \in ciudades1 \Then \existe[unalinea]{c2}{Ciudad}{c2 \in ciudades2 \land c1_0 = c2_0}}
}

\pred{esMatrizCuadrada}{matriz: \matrizEnt}
{
	\paraTodo[unalinea]{i}{\ent}{0 \leq i < |matriz| \implicaLuego \longitud{matriz[i]} = \longitud{matriz}}
}
\pred{existeCamino}{distancias: \matrizEnt, desde: \ent, hasta: \ent} {
	\existe {s} {\TLista{\ent}} {
		secuenciaEnMatriz(s, distancias)  \yLuego caminoValido(s,distancias, desde, hasta)
	}
}
\pred{secuenciaEnMatriz}{secuencia: \TLista{\ent}, matriz: \matrizEnt} {
	\paraTodo {i} {\ent}{enRango(i,secuencia) \implicaLuego 0 \leq secuencia[i] < \longitud{matriz}}
}


\pred{caminoValido}
{camino: \TLista{\ent}, distancias: \matrizEnt, desde: \ent, hasta: \ent}
{
\longitud{camino} \geq 2 \yLuego \\
camino[0] = desde \land  camino[\longitud{camino} -1] = hasta \land \\
\paraTodo[unalinea]{i} {\ent} { 0 \leq i < \longitud{camino} - 1 \\  	\implicaLuego distancias \; [\;camino[i]\;] \; [\;camino[i+1]\;] \neq 0}
}

\pred{enRango}{i: \ent, s: \TLista{\ent}}
{
	0 \leq i < \longitud{s}
}





\section{Demostraciones de correctitud}

En el siguiente apartado, demostraremos la corrección del código solicitado \say{poblaciónTotal} en relación con la especificación proporcionada para dicho problema.  \say{La función poblaciónTotal recibe una lista de ciudades donde al menos una de ellas es grande (es decir, supera los
	50.000 habitantes) y devuelve la cantidad total de habitantes.}



\end{document}
