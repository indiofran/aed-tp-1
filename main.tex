 \documentclass[10pt,a4paper]{article}

\usepackage[spanish,activeacute,es-tabla]{babel}
\usepackage[utf8]{inputenc}
\usepackage{ifthen}
\usepackage{listings}
\usepackage{dsfont}
\usepackage{subcaption}
\usepackage{amsmath}
\usepackage[strict]{changepage}
\usepackage[top=1cm,bottom=2cm,left=1cm,right=1cm]{geometry}%
\usepackage{color}%
\newcommand{\tocarEspacios}{%
	\addtolength{\leftskip}{3em}%
	\setlength{\parindent}{0em}%
}

% Especificacion de procs

\newcommand{\In}{\textsf{in }}
\newcommand{\Out}{\textsf{out }}
\newcommand{\Inout}{\textsf{inout }}

\newcommand{\encabezadoDeProc}[4]{%
% Ponemos la palabrita problema en tt
%  \noindent%
{\normalfont\bfseries\ttfamily proc}%
% Ponemos el nombre del problema
\ %
{\normalfont\ttfamily #2}%
\
% Ponemos los parametros
(#3)%
\ifthenelse{\equal{#4}{}}{}{%
	% Por ultimo, va el tipo del resultado
	\ : #4}
}

\newenvironment{proc}[4][res]{%

% El parametro 1 (opcional) es el nombre del resultado
% El parametro 2 es el nombre del problema
% El parametro 3 son los parametros
% El parametro 4 es el tipo del resultado
% Preambulo del ambiente problema
% Tenemos que definir los comandos requiere, asegura, modifica y aux
\newcommand{\requiere}[2][]{%
{\normalfont\bfseries\ttfamily requiere\;}%
\ifthenelse{\equal{##1}{variaslineas}}{\{%
\begin{adjustwidth}{+5em}{}
	\ensuremath{##2}
\end{adjustwidth}
\}
{\normalfont\bfseries\,\par}%
}
{%
\{\ensuremath{##2}\}%
{\normalfont\bfseries\,\par}%
}
}

\newcommand{\asegura}[2][]{%
{\normalfont\bfseries\ttfamily asegura\;}%
\ifthenelse{\equal{##1}{variaslineas}}{\{%
\begin{adjustwidth}{+5em}{}
	\ensuremath{##2}
\end{adjustwidth}
\}
{\normalfont\bfseries\,\par}%
}
{%
\{\ensuremath{##2}\}%
{\normalfont\bfseries\,\par}%
}
}

\renewcommand{\aux}[4]{%
	{\normalfont\bfseries\ttfamily aux\ }%
		{\normalfont\ttfamily ##1}%
	\ifthenelse{\equal{##2}{}}{}{\ (##2)}\ : ##3\, = \ensuremath{##4}%
	{\normalfont\bfseries\,;\par}%
}
\renewcommand{\pred}[3]{%
{\normalfont\bfseries\ttfamily pred }%
	{\normalfont\ttfamily ##1}%
\ifthenelse{\equal{##2}{}}{}{\ (##2) }%
\{%
\begin{adjustwidth}{+5em}{}
	\ensuremath{##3}
\end{adjustwidth}
\}%
{\normalfont\bfseries\,\par}%
}

\newcommand{\res}{#1}
\vspace{1ex}
\noindent
\encabezadoDeProc{#1}{#2}{#3}{#4}
% Abrimos la llave
\par%
\tocarEspacios
}
{
% Cerramos la llave
\vspace{1ex}
}

\newcommand{\aux}[4]{%
	{\normalfont\bfseries\ttfamily\noindent aux\ }%
		{\normalfont\ttfamily #1}%
	\ifthenelse{\equal{#2}{}}{}{\ (#2)}\ : #3\, = \ensuremath{#4}%
	{\normalfont\bfseries\,;\par}%
}

\newcommand{\pred}[3]{%
{\normalfont\bfseries\ttfamily\noindent pred }%
	{\normalfont\ttfamily #1}%
\ifthenelse{\equal{#2}{}}{}{\ (#2) }%
\{%
\begin{adjustwidth}{+2em}{}
	\ensuremath{#3}
\end{adjustwidth}
\}%
{\normalfont\bfseries\,\par}%
}

% Tipos

\newcommand{\nat}{\ensuremath{\mathds{N}}}
\newcommand{\ent}{\ensuremath{\mathds{Z}}}
\newcommand{\float}{\ensuremath{\mathds{R}}}
\newcommand{\bool}{\ensuremath{\mathsf{Bool}}}
\newcommand{\cha}{\ensuremath{\mathsf{Char}}}
\newcommand{\str}{\ensuremath{\mathsf{String}}}

% Logica

\newcommand{\True}{\ensuremath{\mathrm{true}}}
\newcommand{\False}{\ensuremath{\mathrm{false}}}
\newcommand{\Then}{\ensuremath{\rightarrow}}
\newcommand{\Iff}{\ensuremath{\leftrightarrow}}
\newcommand{\implica}{\ensuremath{\longrightarrow}}
\newcommand{\IfThenElse}[3]{\ensuremath{\mathsf{if}\ #1\ \mathsf{then}\ #2\ \mathsf{else}\ #3\ \mathsf{fi}}}
\newcommand{\yLuego}{\land _L}
\newcommand{\oLuego}{\lor _L}
\newcommand{\implicaLuego}{\implica _L}

\newcommand{\cuantificador}[5]{%
	\ensuremath{(#2 #3: #4)\ (%
		\ifthenelse{\equal{#1}{unalinea}}{
			#5
		}{
			$ % exiting math mode
				\begin{adjustwidth}{+2em}{}
					$#5$%
				\end{adjustwidth}%
			$ % entering math mode
		}
		)}
}

\newcommand{\existe}[4][]{%
	\cuantificador{#1}{\exists}{#2}{#3}{#4}
}
\newcommand{\paraTodo}[4][]{%
	\cuantificador{#1}{\forall}{#2}{#3}{#4}
}

%listas

\newcommand{\TLista}[1]{\ensuremath{seq \langle #1\rangle}}
\newcommand{\lvacia}{\ensuremath{[\ ]}}
\newcommand{\lv}{\ensuremath{[\ ]}}
\newcommand{\longitud}[1]{\ensuremath{|#1|}}
\newcommand{\cons}[1]{\ensuremath{\mathsf{addFirst}}(#1)}
\newcommand{\indice}[1]{\ensuremath{\mathsf{indice}}(#1)}
\newcommand{\conc}[1]{\ensuremath{\mathsf{concat}}(#1)}
\newcommand{\cab}[1]{\ensuremath{\mathsf{head}}(#1)}
\newcommand{\cola}[1]{\ensuremath{\mathsf{tail}}(#1)}
\newcommand{\sub}[1]{\ensuremath{\mathsf{subseq}}(#1)}
\newcommand{\en}[1]{\ensuremath{\mathsf{en}}(#1)}
\newcommand{\cuenta}[2]{\mathsf{cuenta}\ensuremath{(#1, #2)}}
\newcommand{\suma}[1]{\mathsf{suma}(#1)}
\newcommand{\twodots}{\ensuremath{\mathrm{..}}}
\newcommand{\masmas}{\ensuremath{++}}
\newcommand{\matriz}[1]{\TLista{\TLista{#1}}}
\newcommand{\seqchar}{\TLista{\cha}}

\renewcommand{\lstlistingname}{Código}
\lstset{% general command to set parameter(s)
	language=Java,
	morekeywords={endif, endwhile, skip},
	basewidth={0.47em,0.40em},
	columns=fixed, fontadjust, resetmargins, xrightmargin=5pt, xleftmargin=15pt,
	flexiblecolumns=false, tabsize=4, breaklines, breakatwhitespace=false, extendedchars=true,
	numbers=left, numberstyle=\tiny, stepnumber=1, numbersep=9pt,
	frame=l, framesep=3pt,
	captionpos=b,
}

\renewcommand{\wp}[2]{
	\ensuremath{\textit{wp}(\textbf{\lstinline{#1}}, #2)}
}

\newcommand{\hoare}[3]{\ensuremath{\{#1\} \; #2 \; \{#3\}}}

\newcommand{\Def}[1]{\text{\normalfont\ttfamily def}(#1)}
\usepackage{caratula} % Version modificada para usar las macros de algo1 de ~> https://github.com/bcardiff/dc-tex
\usepackage{dirtytalk}
\usepackage{algorithm}
\usepackage{algpseudocode}

% Custom macros
\newcommand{\ciudad}{\ensuremath{\langle \mathsf{String} , \ent \rangle}}
% Custom macros
\newcommand{\matrizEnt}{\TLista{\TLista{\ent}}}

\titulo{Trabajo práctico 1: Especificación y WP}
\subtitulo{En búsqueda del camino}

\fecha{\today}

\materia{Algoritmos Y Estructura de Datos}
\grupo{Grupo QUKDCOGWWYZWJZQBIYWM}

\integrante{Gomez Salaverri, Francisco Nicolas}{550/15}{fsalaverri@dc.uba.ar}
\integrante{Barberón, Federico Joaquín}{112/24}{jfedericobarberonj@gmail.com}
\integrante{Fuertes Vila, Isabel}{474/92}{isblfv@gmail.com}
\integrante{Castelli, Tomás Valentín}{094/23}{tomycastelli@gmail.com}
% Pongan cuantos integrantes quieran

% Declaramos donde van a estar las figuras
% No es obligatorio, pero suele ser comodo
\graphicspath{{../static/}}


% Document
\begin{document}

\maketitle

\section{Problemas}
A continuación, se presentan los procedimientos solicitados en el enunciado del trabajo, junto con las definiciones, funciones auxiliares y predicados que forman parte de la especificación del problema ``En búsqueda del camino'' de este trabajo práctico.

\subsection{Definiciones}
Las siguientes definiciones se utilizan para proporcionar mayor claridad y legibilidad al trabajo.
\medskip

Ciudad:  \ciudad \; siendo $Ciudad_0$ el nombre de la ciudad y a su vez es el identificador unico de las ciudades, basandonos en el chequeo que realiza la catedra en el punto 2.
\medskip

Consideramos como ``caminos válidos'' aquellos que conectan dos o más ciudades distintas, siempre y cuando exista una conexión real entre ellas. Por lo tanto, un camino como $\langle A \rangle$ no es válido, ya que no implica un desplazamiento entre ciudades, tal y como fue aclarado por la catedra. Sin embargo, un camino como $\langle A, B, A \rangle$, donde $A$ y $B$ están conectados, lo consideramos como válido dado que involucra un recorrido efectivo hacia una ciudad diferente $(B)$ antes de regresar a la ciudad inicial $(A)$. En resumen, cualquier camino que implique al menos una conexión entre ciudades distintas y en el que dichas ciudades estén conectadas es considerado válido, incluso si eventualmente regresa al punto de partida.

\subsection{Procedimientos}


\begin{proc}{grandesCiudades}{\In ciudades : \TLista{Ciudad}}{ \TLista{Ciudad}}
	\requiere{todasCiudadesDistintas(ciudades)}
	\asegura[variaslineas]{
		|res| \leq |ciudades| \land \\
		\paraTodo[unalinea] {r}{Ciudad}{r \in res \Iff r \in ciudades \land esCiudadGrande(r)}
	}
\end{proc}

\begin{proc}{sumaDeHabitantes}
	{\In menoresDeCiudades : \TLista{Ciudad}, \In mayoresDeCiudades : \TLista{Ciudad}}{ \TLista{Ciudad}}

	\requiere[variaslineas]{
		|menoresDeCiudades| = |mayoresDeCiudades| \land \\
		todasCiudadesDistintas(menoresDeCiudades) \land \\
		tienenMismosNombres(menoresDeCiudades, mayoresDeCiudades)
	}
	\asegura[variaslineas]{
		|res| = |menoresDeCiudades| \land \\
		tienenMismosNombres(res, menoresDeCiudades) \land \\
		\paraTodo[unalinea]{c}{Ciudad}{c \in res \Then \\ c_1 = cantidadHabitantes(c_0, menoresDeCiudades) + cantidadHabitantes(c_0, mayoresDeCiudades)}
	}
\end{proc}

\begin{proc}{hayCamino}{\In distancias : \matrizEnt, in desde: \ent, hasta: \ent}{ \bool}
	\requiere[variaslineas]{
		0 \leq desde,hasta < \longitud{distancias} \land \\
		esSimetrica(m) \land \\
		tieneDiagonalNula(m) \land \\
		todosPositivos(m)
	}
	\asegura{res = existeCamino(distancias, desde, hasta)}
\end{proc}

\begin{proc}{cantidadCaminosNSaltos}{\Inout conexion : \matrizEnt, \In n: \ent}{}
	\requiere{
		esMatrizConexion(conexion) \land n \geq 1 \land conexion = C_0
	}
	\asegura[variaslineas]{
		\existe[unalinea]{s}{\TLista{\matrizEnt}}{|s| = n \yLuego C_0 = s[0] \land conexion = s[|s| -1] \yLuego \\
			\paraTodo[unalinea]{i}{\ent}{1\leq i < |s| \implicaLuego esProductoDeMatriz(s[i], s[i-1], s[0])} }
	}
\end{proc}

\pagebreak

\begin{proc}{caminoMínimo}
	{\In origen:\ent , \In destino: \ent , \In distancias : \TLista{\TLista{\ent}}}
	{\TLista{\ent}}

	\requiere[variaslineas]{
		0 \leq desde,hasta < \longitud{distancias} \land \\
		esSimetrica(m) \land \\
		tieneDiagonalNula(m) \land \\
		todosPositivos(m)
	}
	\asegura[variaslineas]{
		\neg existeCamino(distancias, origen, destino) \Then res = \langle \rangle \land \\
		existeCamino(distancias, origen, destino) \Then (\paraTodo[unalinea]{i} {\ent}{enRango(i,res) \implicaLuego enRango(res[i], distancias)} \yLuego \\
		caminoValido(res, distancias, origen, destino) \land \\
		\paraTodo{c}{\TLista{\ent}}{
			(\paraTodo[unalinea]{i} {\ent}{enRango(i,c) \implicaLuego enRango(c[i], distancias)} \yLuego caminoValido(c, distancias, origen, destino)) \\
			\implicaLuego distanciaDeCamino(res, distancias) \leq distanciaDeCamino(c, distancias)
		})
	}
\end{proc}


\subsection{Auxiliares}
\aux{cantidadHabitantes}{nombre: \str, ciudades: \TLista{Ciudad}}{\ent}{\\
	\displaystyle\sum_{i = 0}^{|ciudades| - 1} IfThenElse(ciudades[i]_0 = nombre, ciudades[i]_1, 0)
}

\aux{distanciaDeCamino}{c: \TLista{\ent}, distancias: \matrizEnt}{\ent}{
\displaystyle\sum_{i=0}^{|c| - 2} distancias[\;c[i]\;][\;c[i + 1]\;]
}

\subsection{Predicados}
\pred{todasCiudadesDistintas}{ciudades: \TLista{Ciudad}}{
	\paraTodo[unalinea]{i}{\ent}{enRango(i, ciudades) \implicaLuego \neg \existe[unalinea]{j}{\ent}{(enRango(j, ciudades) \land i \neq j) \yLuego ciudades[i]_0 = ciudades[j]_0}}
}

\pred{tienenMismosNombres}{ciudades1: \TLista{Ciudad}, ciudades2: \TLista{Ciudad}}{
	\paraTodo[unalinea]{c1}{Ciudad}{c1 \in ciudades1 \Then \existe[unalinea]{c2}{Ciudad}{c2 \in ciudades2 \land c1_0 = c2_0}}
}

\pred{esCiudadGrande}{ciudad: Ciudad}{ciudad_1 > 50,000}

\pred{esMatrizCuadrada}{matriz: \matrizEnt}
{
	\paraTodo[unalinea]{i}{\ent}{enRango(i, matriz) \implicaLuego \longitud{matriz[i]} = \longitud{matriz}}
}

\pred{esSimetrica}{m: \matrizEnt}{
	esMatrizCuadrada(m) \yLuego \\
	\paraTodo[unalinea]{i}{\ent}{
		\paraTodo[unalinea]{j}{\ent}{
			(enRango(i,m) \land enRango(j,m)) \implicaLuego s[i][j] = s[j][i]
		}
	}
}

\pred{todosPositivos}{m: \matrizEnt}{
\paraTodo[unalinea]{i}{\ent}{
\paraTodo[unalinea]{j}{\ent}{
(enRango(i,m) \land enRango(j,m[i])) \implicaLuego s[i][j] \geq 0
}
}
}

\pred{tieneDiagonalNula}{m: \matrizEnt}{
	\paraTodo{i}{\ent}{enRango(i,m) \implicaLuego m[i][i] = 0}
}

\pred{existeCamino}{distancias: \matrizEnt, desde: \ent, hasta: \ent} {
	\existe {s} {\TLista{\ent}} {
		\paraTodo[unalinea]{i} {\ent}{enRango(i,s) \implicaLuego enRango(s[i], distancias)}  \yLuego caminoValido(s,distancias, desde, hasta)
	}
}

\pred{caminoValido}
{camino: \TLista{\ent}, distancias: \matrizEnt, desde: \ent, hasta: \ent}
{
\longitud{camino} \geq 2 \yLuego \\
camino[0] = desde \land  camino[\longitud{camino} -1] = hasta \land \\
\paraTodo{i} {\ent} { 0 \leq i < \longitud{camino} - 1  	\implicaLuego distancias \; [\;camino[i]\;] \; [\;camino[i+1]\;] \neq 0}
}

\pred{esProductoDeMatriz}{A: \matrizEnt, M1: \matrizEnt, M2: \matrizEnt}
{
	|A| = |M1| \land |M1| = |M2| \land \\
	esMatrizCuadrada(A) \land esMatrizCuadrada(M1) \land esMatrizCuadrada(M2) \yLuego \\
	\paraTodo[unalinea]{i}{\ent}{\paraTodo{j}{\ent}{enRango(j,A) \land enRango(i,A) \implicaLuego \\
			A[i][j] = \sum _{r=0} ^ {|A|-1} M1[i][r] * M2[r][j]
		}
	}
}

\pred{esMatrixConexion}{conexion: \matrizEnt}{
	(esSimetrica(conexion) \land tieneDiagonalNula(conexion)) \yLuego \\
	\paraTodo[unalinea]{i}{\ent}{\paraTodo{j}{\ent}{enRango(i,conexiones) \land \\enRango(j,conexiones)\implicaLuego \\
			conexiones[i][j] = 1 \lor conexiones[i][j] = 0
		}
	}}

\pred{enRango}{i: \ent, s: \TLista{T}}
{
	0 \leq i < \longitud{s}
}


\section{Demostraciones de correctitud}

\subsection{Correctiud}
En el siguiente apartado, demostraremos la corrección del código solicitado \say{poblaciónTotal} en relación con la especificación proporcionada para dicho problema.  \say{La función poblaciónTotal recibe una lista de ciudades donde al menos una de ellas es grande (es decir, supera los 50.000 habitantes) y devuelve la cantidad total de habitantes.}

\begin{lstlisting}
res := 0;
i := 0;
while (i < |ciudades|) do
	res := res + ciudades[i].habitantes;
	i := i + 1
endwhile
\end{lstlisting}

Para probrar al correctitud de este codigo tiene que cumplirse la tripla de Hoare \hoare{P}{S}{Q}, que se prueba mostrando que $P \Then \wp{S}{Q}$

Para esto, podemos pensar al programa $S$ como una sucesión de programas $S1;S2$:
%
\begin{align*}
	S1 & \equiv \lstinline|res := 0; i := 0| \\
	S2 & \equiv \lstinline|ciclo|
\end{align*}

De esta manera, si probamos que

\begin{itemize}
	\item $P \Then \wp{S1}{P_c}$
	\item $P_c \Then \wp{S2}{Q}$
\end{itemize}

entonces, por corolario de monotonía, $P \Then \wp{S}{Q}$, es decir, el programa es correcto.
\bigskip

{\large$P_c \Then \wp{S2}{Q}$}
\medskip

Como $S2$ es un ciclo, no tenemos un axioma que nos permita calcular la \wp{S2}{Q}, sin embargo, podemos probar que la tripla de Hoare \hoare{Pc}{S2}{Q} es correcto mediante el teorema del invariante y el teorema de terminación.
\pagebreak

Según el teorema de invariante, si existe un predicado $I$ tal que.. \par

\begin{enumerate}
	\item $P_c \Longrightarrow I$
	\item \hoare {I \wedge B} {S2} {I}
	\item $(I \wedge  \neg B) \Longrightarrow Q$
\end{enumerate}

Entonces el programa $S2$ es parcialmente correcto con respecto a la especificación.
\bigskip

Para simplificar la escritura, vamos a definir unos predicados:
%
\begin{align*}
	hayCiudadGrande \equiv     & \existe[unalinea]{i}{\ent}{0 \leq i < |ciudades| \yLuego ciudades[i].habitantes > 50000}                                          \\
	habitantesPositivos \equiv & \paraTodo[unalinea]{i}{\ent}{0 \leq i < |ciudades| \implicaLuego ciudades[i].habitantes \geq 0}                                   \\
	ciudadesUnicas \equiv      & (\forall i: \ent)\paraTodo[unalinea]{j}{\ent}{0 \leq i < j < |ciudades| \implicaLuego ciudades[i].nombre \neq ciudades[j].nombre} \\
	C \equiv                   & hayCiudadGrande \land habitantesPositivos
\end{align*}

Sea \( Q\) la poscondición y \( B \) la guarda del bucle `while`.

La relación entre estas condiciones es la siguiente:
%
\begin{align*}
	P \equiv   & \{C \land ciudadesUnicas \}                                                                                                         \\
	P_c \equiv & \{C \land res = 0 \land i = 0\}                                                                                                     \\
	Q   \equiv & \left\{\text{res} = \sum_{i=0}^{|\text{ciudades}|-1} \text{ciudades}[i].\text{habitantes} \right\}                                  \\
	B   \equiv & \{ i < |ciudades| \}                                                                                                                \\
	I   \equiv & \left\{ C \land 0 \leq i \leq |\text{ciudades}| \yLuego \text{res} = \sum_{j=0}^{i-1} \text{ciudades}[j].\text{habitantes} \right\} \\
	f_v \equiv & \left\{ |ciudades| - i \right\}
\end{align*}

Para demostrar que \( P_c \implies I \), verificamos lo siguiente:

\begin{enumerate}
	\item Demostración de $C$:

	      Claramente si vale $P_c$ en particular vale $C$, por lo que $C \Then C \equiv True$.

	\item Demostración de \( 0 \leq i \leq |ciudades| \):

	      Asumiendo que vale el antecedente, se tiene que
	      \[
		      i = 0 \land 0 \leq i \leq |ciudades| \equiv 0 \leq 0 \leq |ciudades|
	      \]

	      Esto es cierto ya que claramente \( 0 \leq |ciudades| \). Por lo tanto, se cumple \( 0 \leq 0 \leq |s| \).

	\item Demostración de la sumatoria en \( I \):

	      En \( P_c \), tenemos que \( \text{res} = 0 \). La expresión para la sumatoria en \( I \) es:

	      \[
		      \text{res} = \sum_{j=0}^{i-1} \text{ciudades}[j].\text{habitantes}
	      \]

	      Cuando \( i = 0 \), la sumatoria va desde \( j = 0 \) hasta \( j = -1 \), lo cual resulta en una sumatoria vacía:

	      \[
		      \sum_{j=0}^{-1} \text{ciudades}[j].\text{habitantes} = 0
	      \]

	      Por lo tanto, \( \text{res} = 0 \), lo cual coincide con la precondición \( \text{res} = 0 \) en \( P_c \).
\end{enumerate}
\pagebreak

Para demostrar que \hoare{I \land B}{S2}{I}:

Sabemos que $\hoare{I \land B}{S2}{I} \iff (I \land B) \implica \wp{S2}{I}$. Entonces:

\begin{align*}
	\wp{S2}{I} & \equiv  \wp{ res := res + ciudades[i].habitantes; i := i + 1}{C \land 0 \leq i \leq |ciudades| \yLuego res = \sum_{j=0}^{i-1} ciudades[j].habitantes}      \\
	           & \equiv  \wp{ res := res + ciudades[i].habitantes}{\wp{i := i + 1}{C \land 0 \leq i \leq |ciudades| \yLuego res = \sum_{j=0}^{i-1} ciudades[j].habitantes}} \\
	           & \equiv  \wp{ res := res + ciudades[i].habitantes}{C \land 0 \leq i+1 \leq |ciudades| \yLuego res = \sum_{j=0}^{i-1 + 1} ciudades[j].habitantes}            \\
	           & \equiv  \wp{ res := res + ciudades[i].habitantes}{C \land 0 \leq i+1 \leq |ciudades| \yLuego res = \sum_{j=0}^{i} ciudades[j].habitantes}                  \\
	           & \equiv  \Def{ciudades[i]} \yLuego C \land -1 \leq i \leq |ciudades| -1 \yLuego res + ciudades[i].habitantes = \sum_{j=0}^{i} ciudades[j].habitantes        \\
	           & \equiv  C \land 0 \leq i < |ciudades| \yLuego res = \sum_{j=0}^{i} ciudades[j].habitantes - ciudades[i].habitantes                                         \\
	           & \equiv  C \land 0 \leq i < |ciudades| \yLuego res = \sum_{j=0}^{i-1} ciudades[j].habitantes
\end{align*}


Entonces $\ensuremath{I \land B}$  \implica \wp{S2}{I}?

\begin{align*}
	{I \land B}  \equiv & C \land 0 \leq i \leq |ciudades|  \yLuego res = \sum_{j=0}^{i-1} ciudades[j].habitantes \land i < |ciudades| \\
	\equiv              & C \land 0 \leq i < |ciudades| \yLuego res = \sum_{j=0}^{i-1} ciudades[j].habitantes
\end{align*}

Que es exactamente \wp{S2}{I}, y por lo tanto $I \land B \Then\wp{S2}{I} \equiv \wp{S2}{I} \Then \wp{S2}{I} \equiv True$ QED.

\bigskip

Ahora solo queda probar que  $(I \land \neg B) \Then Q$

Donde
\begin{align*}
	I \land \neg B & \equiv C \land |ciudades| \leq i \leq |ciudades| \land \text{res} = \sum_{j=0}^{i-1} \text{ciudades}[j].\text{habitantes} \\
	               & \equiv C \land i = |ciudades| \land \text{res} = \sum_{j=0}^{i-1} \text{ciudades}[j].\text{habitantes}                    \\
	               & \equiv C \land \text{res} = \sum_{j=0}^{|ciudades|-1} \text{ciudades}[j].\text{habitantes}                                \\
\end{align*}

Que es exactamente $C \land Q$, y por lo tanto $(I \land \neg B) \Then Q \equiv (C \land Q) \Then Q \equiv True$ QED.
\bigskip

De esta manera, queda demostrado que el programa es parcialmente correcto. Ahora solo queda demostrar que el programa efectivamente termina, en particular, que el ciclo termina en una cantidad finita de iteraciones. Esto se demuestra mediante el teorema de terminación.
\bigskip

Según este teorema, si existe una función $f_v: \mathds{V} \to \ent$ tal que:

\begin{enumerate}
	\item \hoare{I \land B \land f_v = v_0}{S2}{f_v < v_0}
	\item $(I \land f_v \leq 0) \Then \neg B$
\end{enumerate}

entonces la ejecución del ciclo S2 siempre termina.
\bigskip

Por lo cual, probamos si la función variante propuesta $f_v = |ciudades| - i$ cumple estas condiciones.
\bigskip

\begin{enumerate}
	\item \hoare{I \land B \land f_v = v_0}{S2}{f_v < v_0}

	      Sabemos que $\hoare{I \land B \land f_v = v_0}{S}{f_v < v_0} \iff (I \land B \land f_v = v_0) \Then \wp{S2}{f_v < v_0}$, entonces:

	      \begin{align*}
		      \wp{S2}{f_v < v_0} & \equiv \wp{res := res + ciudades[i].habitantes; i := i + 1;}{|ciudades| - i < v_0} \\
		                         & \equiv \wp{res := res + ciudades[i].habitantes}{\wp{i := i + 1}{f_v < v_0}}        \\
		                         & \equiv \wp{res := res + ciudades[i].habitantes}{|ciudades| - i - 1 < v_0}          \\
		                         & \equiv 0 \leq i < |ciduades| \land |ciudades| - i - 1 < v_0
	      \end{align*}

	      Por lo tanto,
	      \begin{align*}
		       & (I \land B \land f_v = v_0) \Then \wp{S2}{f_v < v_0}                                                                                                                                                                                 \\
		       & \equiv ((0 \leq i < |ciudades| \yLuego \underbrace{C \land res = \sum_{j = 0}^{i - 1} ciudades[j].habitantes}_{\text{No me aporta nada en la implicación}}) \land i < |ciudades| \land |ciudades| - i = v_0) \Then \wp{S}{f_v < v_0} \\
		       & \equiv (0 \leq i < |ciuades| \land |ciudades| - i = v_0) \Then (0 \leq i < |ciduades| \land |ciudades| - i - 1 < v_0)                                                                                                                \\
		       & \equiv |ciudades| - i = v_0 \Then |ciudades| - i - 1 < v_0                                                                                                                                                                           \\
		       & \equiv |ciudades| - i - 1 < |ciudades| - i                                                                                                                                                                                           \\
		       & \equiv -1 < 0 \equiv True
	      \end{align*}

	\item $(I \land f_v \leq 0) \Then \neg B$

	      Tenemos que
	      \begin{align*}
		      (I \land f_v \leq 0) \Then \neg B & \equiv ((0 \leq i \leq |ciudades| \yLuego \underbrace{C \land res = \sum_{j = 0}^{i - 1} ciudades[j].habitantes}_{\text{No me aporta nada en la implicación}}) \land |ciudades| - i \leq 0) \Then i \geq |ciudades| \\
		                                        & \equiv (0 \leq i \leq |ciudades| \land |ciudades| \leq i) \Then i \geq |ciduades|                                                                                                                                   \\
		                                        & \equiv i = |ciudades| \Then i \geq |ciudades|                                                                                                                                                                       \\
		                                        & \equiv |ciudades| \geq |ciudades| \equiv True
	      \end{align*}
\end{enumerate}

De esta manera, queda demostrado que el ciclo termina en una cantidad finita de iteraciones, por lo que así, en conjunto con lo probado anteriormente, queda demostrado que la tripla de Hoare \hoare{Pc}{S2}{Q} es \textbf{correcta}.
\bigskip

{\large$P \Then \wp{S1}{P_c}$}
\medskip

Primero calculamos \wp{S1}{Pc}
%
\begin{align*}
	\wp{S1}{P_c} & \equiv \wp{res := 0; i := 0}{C \land res = 0 \land i = 0}      \\
	             & \equiv \wp{res := 0}{\wp{i := 0}{C \land res = 0 \land i = 0}} \\
	             & \equiv \wp{res := 0}{C \land res = 0 \land 0 = 0}              \\
	             & \equiv C \land 0 = 0 \land 0 = 0 \equiv C
\end{align*}

y como $P \Then \wp{S1}{P_c} \equiv (C \land ciudadesUnicas) \Then C \equiv True$ QED.
\bigskip

Luego, como quedó demostrado que

\begin{itemize}
	\item $P \Then \wp{S1}{P_c}$
	\item $\hoare{P_c}{S2}{Q}$
\end{itemize}

por corolario de monotonía $P \Then \wp{S}{Q}$, es decir, la tripla de Hoare \hoare{P}{S}{Q} es \textbf{válida}.

\pagebreak

\subsection{Correctitud b}

Para probar que $res > 50000$ definimos una nueva postcondición $Q'$:

\[Q' \equiv \left\{\text{res} = \sum_{i=0}^{|\text{ciudades}|-1} \text{ciudades}[i].\text{habitantes} \land res > 50000\right\}\]

De esta manera, nos alcanza con probar que la tripla de Hoare \hoare{P}{S}{Q'} es válida.
\bigskip

Como lo único que cambia con respecto al ítem anterior es la postcondición, y solamente está involucrada en el tercer punto de la demostración del teorema del invariante, no es necesario volver a demostrar todo lo demás, ya que es exactamente igual al ítem anterior y, por lo tanto, ya está demostrado.
\medskip

Teniendo en cuenta eso, es suficiente con demostrar únicamente el tercer punto del teorema del invariante: $(I \land \neg B) \Then Q'$
\medskip

Tenemos que
%
\begin{align*}
	I \land \neg B \equiv & C \land 0 \leq i \leq |\text{ciudades}| \yLuego \text{res} = \sum_{j=0}^{i-1} \text{ciudades}[j].\text{habitantes} \land i \geq |ciudades| \\
	\equiv                & C \land |ciudades| \leq i \leq |\text{ciudades}| \yLuego \text{res} = \sum_{j=0}^{i-1} \text{ciudades}[j].\text{habitantes}                \\
	\equiv                & C \land i = |ciudades| \yLuego \text{res} = \sum_{j=0}^{i-1} \text{ciudades}[j].\text{habitantes}                                          \\
	\equiv                & C \land res = \sum_{j=0}^{|ciudades| - 1} \text{ciudades}[j].\text{habitantes}
\end{align*}

Entonces $(I \land \neg B) \Then Q'$?
%
\begin{enumerate}
	\item Demostración de la sumatoria:

	      Vemos que claramente si vale $I \land \neg B$, en particular vale $res = \sum_{j=0}^{|ciudades| - 1} \text{ciudades}[j].\text{habitantes}$, que es la primera parte de la $Q'$

	\item Demostración de $res > 50000$:

	      Si vale $I \land \neg B$, entonces vale $C$ (existe una ciudad con más de 50000 habitantes y todas tienen una cantidad positiva de habitantes) y $res$ es la suma de los habitantes de todas las ciudades, por lo que claramente $res > 50000$.
\end{enumerate}

Por lo tanto $(I \land \neg B) \Then Q'$ es verdadero.
\bigskip

De esta manera, en conjunto con las demostraciónes del ítem anterior, la tripla de Hoare \hoare{P}{S}{Q'} es válida y por lo tanto queda probado que el resultado devuelto por el programa efectivamente es mayor a 50000.
\end{document}
