 \documentclass[10pt,a4paper]{article}

\usepackage[spanish,activeacute,es-tabla]{babel}
\usepackage[utf8]{inputenc}
\usepackage{ifthen}
\usepackage{listings}
\usepackage{dsfont}
\usepackage{subcaption}
\usepackage{amsmath}
\usepackage[strict]{changepage}
\usepackage[top=1cm,bottom=2cm,left=1cm,right=1cm]{geometry}%
\usepackage{color}%
\newcommand{\tocarEspacios}{%
	\addtolength{\leftskip}{3em}%
	\setlength{\parindent}{0em}%
}

% Especificacion de procs

\newcommand{\In}{\textsf{in }}
\newcommand{\Out}{\textsf{out }}
\newcommand{\Inout}{\textsf{inout }}

\newcommand{\encabezadoDeProc}[4]{%
% Ponemos la palabrita problema en tt
%  \noindent%
{\normalfont\bfseries\ttfamily proc}%
% Ponemos el nombre del problema
\ %
{\normalfont\ttfamily #2}%
\
% Ponemos los parametros
(#3)%
\ifthenelse{\equal{#4}{}}{}{%
	% Por ultimo, va el tipo del resultado
	\ : #4}
}

\newenvironment{proc}[4][res]{%

% El parametro 1 (opcional) es el nombre del resultado
% El parametro 2 es el nombre del problema
% El parametro 3 son los parametros
% El parametro 4 es el tipo del resultado
% Preambulo del ambiente problema
% Tenemos que definir los comandos requiere, asegura, modifica y aux
\newcommand{\requiere}[2][]{%
{\normalfont\bfseries\ttfamily requiere\;}%
\ifthenelse{\equal{##1}{variaslineas}}{\{%
\begin{adjustwidth}{+5em}{}
	\ensuremath{##2}
\end{adjustwidth}
\}
{\normalfont\bfseries\,\par}%
}
{%
\{\ensuremath{##2}\}%
{\normalfont\bfseries\,\par}%
}
}

\newcommand{\asegura}[2][]{%
{\normalfont\bfseries\ttfamily asegura\;}%
\ifthenelse{\equal{##1}{variaslineas}}{\{%
\begin{adjustwidth}{+5em}{}
	\ensuremath{##2}
\end{adjustwidth}
\}
{\normalfont\bfseries\,\par}%
}
{%
\{\ensuremath{##2}\}%
{\normalfont\bfseries\,\par}%
}
}

\renewcommand{\aux}[4]{%
	{\normalfont\bfseries\ttfamily aux\ }%
		{\normalfont\ttfamily ##1}%
	\ifthenelse{\equal{##2}{}}{}{\ (##2)}\ : ##3\, = \ensuremath{##4}%
	{\normalfont\bfseries\,;\par}%
}
\renewcommand{\pred}[3]{%
{\normalfont\bfseries\ttfamily pred }%
	{\normalfont\ttfamily ##1}%
\ifthenelse{\equal{##2}{}}{}{\ (##2) }%
\{%
\begin{adjustwidth}{+5em}{}
	\ensuremath{##3}
\end{adjustwidth}
\}%
{\normalfont\bfseries\,\par}%
}

\newcommand{\res}{#1}
\vspace{1ex}
\noindent
\encabezadoDeProc{#1}{#2}{#3}{#4}
% Abrimos la llave
\par%
\tocarEspacios
}
{
% Cerramos la llave
\vspace{1ex}
}

\newcommand{\aux}[4]{%
	{\normalfont\bfseries\ttfamily\noindent aux\ }%
		{\normalfont\ttfamily #1}%
	\ifthenelse{\equal{#2}{}}{}{\ (#2)}\ : #3\, = \ensuremath{#4}%
	{\normalfont\bfseries\,;\par}%
}

\newcommand{\pred}[3]{%
{\normalfont\bfseries\ttfamily\noindent pred }%
	{\normalfont\ttfamily #1}%
\ifthenelse{\equal{#2}{}}{}{\ (#2) }%
\{%
\begin{adjustwidth}{+2em}{}
	\ensuremath{#3}
\end{adjustwidth}
\}%
{\normalfont\bfseries\,\par}%
}

% Tipos

\newcommand{\nat}{\ensuremath{\mathds{N}}}
\newcommand{\ent}{\ensuremath{\mathds{Z}}}
\newcommand{\float}{\ensuremath{\mathds{R}}}
\newcommand{\bool}{\ensuremath{\mathsf{Bool}}}
\newcommand{\cha}{\ensuremath{\mathsf{Char}}}
\newcommand{\str}{\ensuremath{\mathsf{String}}}

% Logica

\newcommand{\True}{\ensuremath{\mathrm{true}}}
\newcommand{\False}{\ensuremath{\mathrm{false}}}
\newcommand{\Then}{\ensuremath{\rightarrow}}
\newcommand{\Iff}{\ensuremath{\leftrightarrow}}
\newcommand{\implica}{\ensuremath{\longrightarrow}}
\newcommand{\IfThenElse}[3]{\ensuremath{\mathsf{if}\ #1\ \mathsf{then}\ #2\ \mathsf{else}\ #3\ \mathsf{fi}}}
\newcommand{\yLuego}{\land _L}
\newcommand{\oLuego}{\lor _L}
\newcommand{\implicaLuego}{\implica _L}

\newcommand{\cuantificador}[5]{%
	\ensuremath{(#2 #3: #4)\ (%
		\ifthenelse{\equal{#1}{unalinea}}{
			#5
		}{
			$ % exiting math mode
				\begin{adjustwidth}{+2em}{}
					$#5$%
				\end{adjustwidth}%
			$ % entering math mode
		}
		)}
}

\newcommand{\existe}[4][]{%
	\cuantificador{#1}{\exists}{#2}{#3}{#4}
}
\newcommand{\paraTodo}[4][]{%
	\cuantificador{#1}{\forall}{#2}{#3}{#4}
}

%listas

\newcommand{\TLista}[1]{\ensuremath{seq \langle #1\rangle}}
\newcommand{\lvacia}{\ensuremath{[\ ]}}
\newcommand{\lv}{\ensuremath{[\ ]}}
\newcommand{\longitud}[1]{\ensuremath{|#1|}}
\newcommand{\cons}[1]{\ensuremath{\mathsf{addFirst}}(#1)}
\newcommand{\indice}[1]{\ensuremath{\mathsf{indice}}(#1)}
\newcommand{\conc}[1]{\ensuremath{\mathsf{concat}}(#1)}
\newcommand{\cab}[1]{\ensuremath{\mathsf{head}}(#1)}
\newcommand{\cola}[1]{\ensuremath{\mathsf{tail}}(#1)}
\newcommand{\sub}[1]{\ensuremath{\mathsf{subseq}}(#1)}
\newcommand{\en}[1]{\ensuremath{\mathsf{en}}(#1)}
\newcommand{\cuenta}[2]{\mathsf{cuenta}\ensuremath{(#1, #2)}}
\newcommand{\suma}[1]{\mathsf{suma}(#1)}
\newcommand{\twodots}{\ensuremath{\mathrm{..}}}
\newcommand{\masmas}{\ensuremath{++}}
\newcommand{\matriz}[1]{\TLista{\TLista{#1}}}
\newcommand{\seqchar}{\TLista{\cha}}

\renewcommand{\lstlistingname}{Código}
\lstset{% general command to set parameter(s)
	language=Java,
	morekeywords={endif, endwhile, skip},
	basewidth={0.47em,0.40em},
	columns=fixed, fontadjust, resetmargins, xrightmargin=5pt, xleftmargin=15pt,
	flexiblecolumns=false, tabsize=4, breaklines, breakatwhitespace=false, extendedchars=true,
	numbers=left, numberstyle=\tiny, stepnumber=1, numbersep=9pt,
	frame=l, framesep=3pt,
	captionpos=b,
}

\renewcommand{\wp}[2]{
	\ensuremath{\textit{wp}(\textbf{\lstinline{#1}}, #2)}
}

\newcommand{\hoare}[3]{\ensuremath{\{#1\} \; #2 \; \{#3\}}}

\newcommand{\Def}[1]{\text{\normalfont\ttfamily def}(#1)}
\usepackage{caratula} % Version modificada para usar las macros de algo1 de ~> https://github.com/bcardiff/dc-tex
\usepackage{dirtytalk}
\usepackage{algorithm}
\usepackage{algpseudocode}

% Custom macros
\newcommand{\ciudad}{\ensuremath{\langle \mathsf{String} , \ent \rangle}}
% Custom macros
\newcommand{\matrizEnt}{\TLista{\TLista{\ent}}}

\titulo{Trabajo práctico 1: Especificación y WP}
\subtitulo{En búsqueda del camino}

\fecha{\today}

\materia{Algoritmos Y Estructura de Datos}
\grupo{Grupo QUKDCOGWWYZWJZQBIYWM}

\integrante{Gomez Salaverri, Francisco Nicolas}{550/15}{fsalaverri@dc.uba.ar}
\integrante{Barberón, Federico Joaquín}{112/24}{jfedericobarberonj@gmail.com}
\integrante{Fuertes Vila, Isabel}{474/92}{isblfv@gmail.com}
\integrante{Castelli, Tomás Valentín}{094/23}{tomycastelli@gmail.com}
% Pongan cuantos integrantes quieran

% Declaramos donde van a estar las figuras
% No es obligatorio, pero suele ser comodo
\graphicspath{{../static/}}


% Document
\begin{document}

\maketitle

\section{Problemas}
A continuación, se presentan los procedimientos solicitados en el enunciado del trabajo, junto con las definiciones, funciones auxiliares y predicados que forman parte de la especificación del problema ``En búsqueda del camino'' de este trabajo práctico.

\subsection{Definiciones}
Las siguientes definiciones se utilizan para proporcionar mayor claridad y legibilidad al trabajo.

Ciudad:  \ciudad \; siendo $Ciudad_0$ el nombre de la ciudad y a su vez es el identificador unico de las ciudades, basandonos en el chequeo que realiza la catedra en el punto 2.


\subsection{Procedimientos}


\begin{proc}{grandesCiudades}{\In ciudades : \TLista{Ciudad}}{ \TLista{Ciudad}}
	\requiere{todasCiudadesDistintas(ciudades)}
	\asegura[variaslineas]{
		|res| \leq |ciudades| \land \\
		\paraTodo[unalinea] {r}{Ciudad}{r \in res \Iff r \in ciudades \land esCiudadGrande(r)}
	}
\end{proc}

\begin{proc}{sumaDeHabitantes}
	{\In menoresDeCiudades : \TLista{Ciudad}, \In mayoresDeCiudades : \TLista{Ciudad}}{ \TLista{Ciudad}}

	\requiere[variaslineas]{
		|menoresDeCiudades| = |mayoresDeCiudades| \land \\
		todasCiudadesDistintas(menoresDeCiudades) \land \\
		tienenMismosNombres(menoresDeCiudades, mayoresDeCiudades)
	}
	\asegura[variaslineas]{
		|res| = |menoresDeCiudades| \land \\
		tienenMismosNombres(res, menoresDeCiudades) \land \\
		\paraTodo[unalinea]{c}{Ciudad}{c \in res \Then \\ c_1 = cantidadHabitantes(c_0, menoresDeCiudades) + cantidadHabitantes(c_0, mayoresDeCiudades)}
	}
\end{proc}

\begin{proc}{hayCamino}{\In distancias : \matrizEnt, in desde: \ent, hasta: \ent}{ \bool}
	\requiere{ 0 \leq desde,hasta < \longitud{distancias} \wedge esMatrizCuadrada(distancias)}
	\asegura{res = \True \Iff existeCamino(distancias, desde, hasta)}
\end{proc}

\begin{proc}{cantidadCaminosNSaltos}{\Inout conexion : \matrizEnt, \In n: \ent}{}
	\requiere{esMatrizConexion(conexion) \land conexion = C_0 \land n \geq 1 }
	\asegura[variaslineas]{
		\existe[unalinea]{s}{\TLista{\matrizEnt}}{|s| = n \yLuego C_0 = s[0] \land conexion = s[|s| -1] \yLuego \\
			\paraTodo[unalinea]{i}{\ent}{0\leq i < |s|-1 \implicaLuego esMatrizMultiplo(s[i], s[i+1])} }
	}
\end{proc}


\begin{proc}{caminoMínimo}
	{\In origen:\ent , \In destino: \ent , \In distancias : \TLista{\TLista{\ent}}}
	{\TLista{\ent}}

	\requiere{0 \leq origen, destino < |distancias| \land esMatrizCuadrada(distancias)}
	\asegura[variaslineas]{
		\neg existeCamino(distancias, origen, destino) \Then res = \langle \rangle \land \\
		existeCamino(distancias, origen, destino) \Then (secuenciaEnMatriz(res, distancias) \yLuego \\
		caminoValido(res, distancias, origen, destino) \land \\
		\paraTodo{c}{\TLista{\ent}}{
			(secuenciaEnMatriz(c, distancias) \yLuego caminoValido(c, distancias, origen, destino)) \\
			\implicaLuego distanciaDeCamino(res, distancias) \leq distanciaDeCamino(c, distancias)
		})
	}
\end{proc}


\subsection{Auxiliares}
\aux{cantidadHabitantes}{nombre: \str, ciudades: \TLista{Ciudad}}{\ent}{\\
	\displaystyle\sum_{i = 0}^{|ciudades| - 1} IfThenElse(ciudades[i]_0 = nombre, ciudades[i]_1, 0)
}

\aux{distanciaDeCamino}{c: \TLista{\ent}, distancias: \matrizEnt}{\ent}{
\displaystyle\sum_{i=0}^{|c| - 2} distancias[\;c[i]\;][\;c[i + 1]\;]
}

\subsection{Predicados}
\pred{todasCiudadesDistintas}{ciudades: \TLista{Ciudad}}{
	\paraTodo[unalinea]{i}{\ent}{enRango(i, ciudades) \implicaLuego \neg \existe[unalinea]{j}{\ent}{(enRango(j, ciudades) \land i \neq j) \yLuego ciudades[i]_0 = ciudades[j]_0}}
}

\pred{tienenMismosNombres}{ciudades1: \TLista{Ciudad}, ciudades2: \TLista{Ciudad}}{
	\paraTodo[unalinea]{c1}{Ciudad}{c1 \in ciudades1 \Then \existe[unalinea]{c2}{Ciudad}{c2 \in ciudades2 \land c1_0 = c2_0}}
}

\pred{esCiudadGrande}{ciudad: Ciudad}{ciudad_1 > 50,000}

\pred{esMatrizCuadrada}{matriz: \matrizEnt}
{
	\paraTodo[unalinea]{i}{\ent}{enRango(i, matriz) \implicaLuego \longitud{matriz[i]} = \longitud{matriz}}
}
\pred{existeCamino}{distancias: \matrizEnt, desde: \ent, hasta: \ent} {
	\existe {s} {\TLista{\ent}} {
		secuenciaEnMatriz(s, distancias)  \yLuego caminoValido(s,distancias, desde, hasta)
	}
}
\pred{secuenciaEnMatriz}{secuencia: \TLista{\ent}, matriz: \matrizEnt} {
	\paraTodo {i} {\ent}{enRango(i,secuencia) \implicaLuego enRango(secuencia[i], matriz)}
}


\pred{caminoValido}
{camino: \TLista{\ent}, distancias: \matrizEnt, desde: \ent, hasta: \ent}
{
\longitud{camino} \geq 2 \yLuego \\
camino[0] = desde \land  camino[\longitud{camino} -1] = hasta \land \\
\paraTodo{i} {\ent} { 0 \leq i < \longitud{camino} - 1  	\implicaLuego distancias \; [\;camino[i]\;] \; [\;camino[i+1]\;] \neq 0}
}

\pred{esMatrizMultiplo}{A: \matrizEnt, B: \matrizEnt}
{
	|A| = |B| \land esMatrizCuadrada(A) \land esMatrizCuadrada(B) \yLuego \\
	\paraTodo[unalinea]{i}{\ent}{\paraTodo{j}{\ent}{enRango(j,A) \land enRango(i,A) \implicaLuego \\
			B[i][j] = \sum _{r=0} ^ {|A|-1} A[i][r] * A[r][j]
		}
	}
}

\pred{esMatrixConexion}{conexion: \matrizEnt}{
	esMatrizCuadrada(conexion) \yLuego \\
	\paraTodo[unalinea]{i}{\ent}{\paraTodo{j}{\ent}{enRango(i,conexiones) \land \\enRango(j,conexiones)\implicaLuego \\
			conexiones[i][j] = 1 \lor conexiones[i][j] = 0
		}
	}}

\pred{enRango}{i: \ent, s: \TLista{T}}
{
	0 \leq i < \longitud{s}
}





\section{Demostraciones de correctitud}

\subsection{Correctiud}
En el siguiente apartado, demostraremos la corrección del código solicitado \say{poblaciónTotal} en relación con la especificación proporcionada para dicho problema.  \say{La función poblaciónTotal recibe una lista de ciudades donde al menos una de ellas es grande (es decir, supera los 50.000 habitantes) y devuelve la cantidad total de habitantes.}

\begin{lstlisting}
res := 0;
i := 0;
while (i < |ciudades|) do
	res := res + ciudades[i].habitantes;
	i := i + 1
endwhile
\end{lstlisting}

% \begin{algorithm}
% 	\caption{Sumar habitantes de ciudades}
% 	\begin{algorithmic}[1]
% 		\State $res \gets 0$
% 		\State $i \gets 0$
% 		\While{$i < ciudades.length$}
% 		\State $res \gets res + ciudades[i].habitantes$
% 		\State $i \gets i + 1$
% 		\EndWhile
% 	\end{algorithmic}
% \end{algorithm}

Para probrar al correctitud de este codigo tiene que cumplirse la tripla de Hoare \hoare {P}{S}{Q}

Según el teorema de invariante, si existe un predicado $I$ tal que.. \par

\begin{enumerate}
	\item $P_c \Longrightarrow I$
	\item \hoare {I \wedge B} {S} {I}
	\item $(I \wedge  \neg B) \Longrightarrow Q_c$
\end{enumerate}

Sea \( P \) la precondición y \( B \) la guarda del bucle `while`.

La relación entre estas condiciones es la siguiente:

\begin{align*}
	P_c & \equiv \{ \text{res} = 0 \land i = 0 \}                                                                                   \\
	Q_c & \equiv \left\{ \text{res} = \sum_{i=0}^{|\text{ciudades}|-1} \text{ciudades}[i].\text{habitantes} \right\}                \\
	B   & \equiv \{ i < |ciudades| \}                                                                                               \\
	I   & \equiv \left\{ 0 \leq i \leq |\text{s}| \land \text{res} = \sum_{j=0}^{i-1} \text{ciudades}[j].\text{habitantes} \right\} \\
	f_v & \equiv \left\{ |ciudades| - i \right\}
\end{align*}

Para demostrar que \( P_c \implies I \), verificamos lo siguiente:

\begin{enumerate}
	\item Demostración de \( 0 \leq i \leq |s| \):

	      \[
		      0 \leq i \leq |s| \equiv 0 \leq 0 \leq |s|
	      \]

	      Esto es cierto ya que \( i = 0 \) en la precondición \( P_c \), y claramente \( 0 \leq |s| \), donde \( |s| \) es la longitud del array de ciudades. Por lo tanto, se cumple \( 0 \leq 0 \leq |s| \).

	\item Demostración de la sumatoria en \( I \):

	      En \( P_c \), tenemos que \( \text{res} = 0 \). La expresión para la sumatoria en \( I \) es:

	      \[
		      \text{res} = \sum_{j=0}^{i-1} \text{ciudades}[j].\text{habitantes}
	      \]

	      Cuando \( i = 0 \), la sumatoria va desde \( j = 0 \) hasta \( j = -1 \), lo cual resulta en una sumatoria vacía:

	      \[
		      \sum_{j=0}^{-1} \text{ciudades}[j].\text{habitantes} = 0
	      \]

	      Por lo tanto, \( \text{res} = 0 \), lo cual coincide con la precondición \( \text{res} = 0 \) en \( P_c \).
\end{enumerate}

Para demostrar que \hoare{I \land B}{S}{I}:

Sabemos que $\hoare{I \land B}{S}{I} \iff (I \land B) \implica \wp{S}{I}$. Entonces:

\begin{align*}
	\wp{S}{I} & \equiv  \wp{ res := res + ciudades[i].habitantes; i := i + 1}{0 \leq i \leq |ciudades| \yLuego res = \sum_{j=0}^{i-1} ciudades[j].habitantes}      \\
	          & \equiv  \wp{ res := res + ciudades[i].habitantes}{\wp{i := i + 1}{0 \leq i \leq |ciudades| \yLuego res = \sum_{j=0}^{i-1} ciudades[j].habitantes}} \\
	          & \equiv  \wp{ res := res + ciudades[i].habitantes}{0 \leq i+1 \leq |ciudades| \yLuego res = \sum_{j=0}^{i-1 + 1} ciudades[j].habitantes}            \\
	          & \equiv  \wp{ res := res + ciudades[i].habitantes}{0 \leq i+1 \leq |ciudades| \yLuego res = \sum_{j=0}^{i} ciudades[j].habitantes}                  \\
	          & \equiv  \Def{ciudades[i]} \yLuego -1 \leq i \leq |ciudades| -1 \yLuego res + ciudades[i].habitantes = \sum_{j=0}^{i} ciudades[j].habitantes        \\
	          & \equiv  0 \leq i \leq |ciudades| \yLuego res = \sum_{j=0}^{i} ciudades[j].habitantes - ciudades[i].habitantes                                      \\
	          & \equiv  0 \leq i \leq |ciudades|  \yLuego res = \sum_{j=0}^{i-1} ciudades[j].habitantes
\end{align*}


Entonces $\ensuremath{I \land B}$  \implica \wp{S}{I}?

\[{I \land B}  \equiv  0 \leq i \leq |ciudades|  \yLuego res = \sum_{j=0}^{i-1} ciudades[j].habitantes \]

Que es exactamente \wp{S}{I}.
\bigskip

Ahora solo queda probar que  $(I \land \neg B) \Then Q_c$

Donde
\begin{align*}
	I \land \neg B & \equiv |s| \leq i \leq |s| \land \text{res} = \sum_{j=0}^{i-1} \text{ciudades}[j].\text{habitantes} \\
	               & \equiv i = |s| \land \text{res} = \sum_{j=0}^{i-1} \text{ciudades}[j].\text{habitantes}             \\
	               & \equiv \text{res} = \sum_{j=0}^{|s|-1} \text{ciudades}[j].\text{habitantes} \equiv Q_c              \\
\end{align*}

De esta manera, queda demostrado que el programa es parcialmente correcto. Ahora solo queda demostrar que el programa efectivamente termina, en particular, que el ciclo termina en una cantidad finita de iteraciones. Esto se demuestra mediante el teorema de terminación.
\bigskip

Según este teorema, si existe una función $f_v: \mathds{V} \to \ent$ tal que:

\begin{enumerate}
	\item \hoare{I \land B \land f_v = v_0}{S}{f_v < v_0}
	\item $(I \land f_v \leq 0) \Then \neg B$
\end{enumerate}

entonces la ejecución del ciclo \lstinline|while B do S endwhile| siempre termina.
\bigskip

Por lo cual, probamos si la función variante propuesta $f_v = |ciudades| - i$ cumple estas condiciones.
\bigskip

\begin{enumerate}
	\item \hoare{I \land B \land f_v = v_0}{S}{f_v < v_0}

	      Sabemos que $\hoare{I \land B \land f_v = v_0}{S}{f_v < v_0} \iff (i \land B \land f_v = v_0) \Then \wp{S}{f_v < v_0}$, entonces:

	      \begin{align*}
		      \wp{S}{f_v < v_0} & \equiv \wp{res := res + ciudades[i].habitantes; i := i + 1;}{|ciudades| - i < v_0} \\
		                        & \equiv \wp{res := res + ciudades[i].habitantes}{\wp{i := i + 1}{f_v < v_0}}        \\
		                        & \equiv \wp{res := res + ciudades[i].habitantes}{|ciudades| - i - 1 < v_0}          \\
		                        & \equiv 0 \leq i < |ciduades| \land |ciudades| - i - 1 < v_0
	      \end{align*}

	      Por lo tanto,
	      \begin{align*}
		       & (I \land B \land f_v = v_0) \Then \wp{S}{f_v < v_0}                                                                                                                                                                          \\
		       & \equiv ((0 \leq i < |ciudades| \yLuego \underbrace{res = \sum_{j = 0}^{i - 1} ciudades[j].habitantes}_{\text{No me aporta nada en la implicación}}) \land i < |ciudades| \land |ciudades| - i = v_0) \Then \wp{S}{f_v < v_0} \\
		       & \equiv (0 \leq i < |ciuades| \land |ciudades| - i = v_0) \Then (0 \leq i < |ciduades| \land |ciudades| - i - 1 < v_0)                                                                                                        \\
		       & \equiv |ciudades| - i = v_0 \Then |ciudades| - i - 1 < v_0                                                                                                                                                                   \\
		       & \equiv |ciudades| - i - 1 < |ciudades| - i                                                                                                                                                                                   \\
		       & \equiv -1 < 0 \equiv True
	      \end{align*}

	\item $(I \land f_v \leq 0) \Then \neg B$

	      Tenemos que
	      \begin{align*}
		      (I \land f_v \leq 0) \Then \neg B & \equiv ((0 \leq i \leq |ciudades| \yLuego \underbrace{res = \sum_{j = 0}^{i - 1} ciudades[j].habitantes}_{\text{No me aporta nada en la implicación}}) \land |ciudades| - i \leq 0) \Then i \geq |ciudades| \\
		                                        & \equiv (0 \leq i \leq |ciudades| \land |ciudades| \leq i) \Then i \geq |ciduades|                                                                                                                           \\
		                                        & \equiv i = |ciudades| \Then i \geq |ciudades|                                                                                                                                                               \\
		                                        & \equiv |ciudades| \geq |ciudades| \equiv True
	      \end{align*}
\end{enumerate}

De esta manera, queda demostrado que el ciclo termina en una cantidad finita de iteraciones, por lo que así, en conjunto con lo probado anteriormente, queda demostrado que la implementación es \textbf{correcta} con respecto a la especificación.

\subsection{Correctitud b}
Después de verificar la corrección del código anterior, podemos observar que la precondición implica nuestra WP calculada previamente. Esto, a su vez, demuestra, según la definición de precondición:
\medskip

Definición: \textit{Una precondición es una condición sobre los argumentos que el programador asume como verdadera. Especifica lo que la función requiere para llevar a cabo su tarea correctamente.}
\medskip

En este caso, la precondición establece que
\begin{align*}
	 & \existe[unalinea]{i}{\ent}{0 \leq i < |ciudades| \yLuego ciudades[i].habitantes > 50.000} \land \\
	 & \paraTodo[unalinea]{i}{\ent}{0 \leq i < |ciudades| \implicaLuego ciudades[i].habitantes \geq 0}
\end{align*}
Esto nos asegura que siempre se devolverá un resultado mayor a 50,000, ya que es un requisito impuesto por la precondición.
\medskip

Por lo tanto, nuestra implementación, que sigue el procedimiento descrito, suma a res los habitantes de una ciudad. Según la precondición, debe existir al menos una ciudad cuyos habitantes sean más de 50,000 y todas deben tener una cantidad positiva de habitantes. Esto garantiza que el valor de res será mayor a 50,000, ya que siempre existe al menos una ciudad que cumple con dicha condición.

\end{document}
