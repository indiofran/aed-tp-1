\documentclass[10pt,a4paper]{article}

\input{AEDmacros}
\usepackage{caratula} % Version modificada para usar las macros de algo1 de ~> https://github.com/bcardiff/dc-tex
\usepackage{dirtytalk}

% Custom macros
\newcommand{\ciudad}{\ensuremath{\langle \mathsf{String} , \ent \rangle}}

\titulo{Trabajo práctico 1: Especificaci´on y WP}
\subtitulo{En búsqueda del camino}

\fecha{\today}

\materia{Algoritmos Y Estructura de Datos}
\grupo{Grupo QUKDCOGWWYZWJZQBIYWM}

\integrante{Gomez Salaverri, Francisco Nicolas}{550/15}{fsalaverri@dc.uba.ar}
\integrante{Apellido, Nombre2}{002/01}{email2@dominio.com}
\integrante{Apellido, Nombre3}{003/01}{email3@dominio.com}
\integrante{Apellido, Nombre4}{004/01}{email4@dominio.com}
% Pongan cuantos integrantes quieran

% Declaramos donde van a estar las figuras
% No es obligatorio, pero suele ser comodo
\graphicspath{{../static/}}


% Document
\begin{document}

\maketitle

\section{Problemas}
A continuación, se presentan los procedimientos solicitados en el enunciado del trabajo, junto con las definiciones, funciones auxiliares y predicados que forman parte de la especificación del problema 'En búsqueda del camino' de este trabajo práctico.

\subsection{Definiciones}
Las siguientes definiciones se utilizan para proporcionar mayor claridad y legibilidad al trabajo.

Ciudad:   \ciudad


\subsection{Procedimientos}


\begin{proc}{grandesCiudades}{\In ciudades : \TLista{Ciudad}}{ \TLista{Ciudad}}
	\requiere{expresionBooleana1}
	\asegura{expresionBooleana2}
\end{proc}

\begin{proc}{sumaDeHabitantes}{\In menoresDeCiudades : \TLista{Ciudad}, \In mayoresDeCiudades : \TLista{Ciudad}}{ \TLista{Ciudad}}
	\requiere{expresionBooleana1}
	\asegura{expresionBooleana2}
\end{proc}

\begin{proc}{hayCamino}{\In distancias : \TLista{\TLista{\ent}}, in desde: \ent, hasta: \ent}{ \bool}
	\requiere{expresionBooleana1}
	\asegura{expresionBooleana2}
\end{proc}

\begin{proc}{cantidadCaminosNSaltos}{\Inout conexion : \TLista{\TLista{\ent}}, \In n: \ent}{}
	\requiere{expresionBooleana1}
	\asegura{expresionBooleana2}
\end{proc}


\begin{proc}{caminoMínimo}{\In origen:\ent , \In destines: \ent , \In distancias : \TLista{\TLista{\ent}}}{ \TLista{\ent}}
	\requiere{expresionBooleana1}
	\asegura{expresionBooleana2}
\end{proc}

\subsection{Auxiliares}
\aux{auxiliar1}{parametros}{tipoRes}{expresion}

\subsection{Predicados}
\pred{pred1}{parametros}{expresion} 

\section{Demostraciones de correctitud}

En el siguiente apartado, demostraremos la corrección del código solicitado \say{poblaciónTotal} en relación con la especificación proporcionada para dicho problema.  \say{La función poblaciónTotal recibe una lista de ciudades donde al menos una de ellas es grande (es decir, supera los
50.000 habitantes) y devuelve la cantidad total de habitantes.}

\end{document}